\documentclass[10pt,a4paper]{article}
\usepackage[utf8]{inputenc}
\usepackage[english]{babel}
\usepackage{amsmath}
\usepackage{amsthm}
\usepackage{amsfonts}
\usepackage{amssymb}
\usepackage{makeidx}
\usepackage{graphicx}

\newtheorem{theorem}{Theorem}[section]
\newtheorem{example}{Example}[section]
\newtheorem{claim}[theorem]{Claim}
\newtheorem{proposition}[theorem]{Proposition}
\newtheorem{lemma}[theorem]{Lemma}
\newtheorem{corollary}[theorem]{Corollary}
\newtheorem{conjecture}[theorem]{Conjecture}
\newtheorem{definition}[theorem]{Definition}
\numberwithin{equation}{subsection}
\newtheorem{remark}{Remark}


\usepackage{amsmath}
\DeclareMathOperator*{\argmax}{arg\,max}
\DeclareMathOperator*{\argmin}{arg\,min}

\author{Lee De Zhang}
\title{Measure Theory, Limit Theorems, and Random Variables}
\begin{document}

\maketitle
\section{Introduction}
This chapter outlines the tools necessary for
a more rigorous study of probability. 

\section{Measure Theory}

An algebra $\mathcal{F}$ is said to be a $\sigma$-algebra
if
\begin{enumerate}
	\item It is closed under complement, i.e., $A \in \mathcal{F} \Rightarrow A^c \in \mathcal{F}$.
	\item Closed under countable union, i.e., $A_i \in \mathcal{F} \Rightarrow \cup_i A_i \in \mathcal{F}$ for
	countable number of $A_i$. 
\end{enumerate}
\begin{remark}
	Countable refers to finite or countably infinite.
	Some textbooks also state closed under countable 
	intersection, but it is easy to derive that 
	from the above conditions. 
\end{remark}.

We will now state the definition of a probability space

\begin{definition}[\emph{Probability Space}]

A probability space is the triple $(\Omega, \mathcal{F}, P)$,
such that $\Omega$ is the support or set of all possible
outcomes (sample space), $\mathcal{F}$ is a set of 
events, and $P:\mathcal{F}\rightarrow[0,1]$ which assigns
probabilities to items in $\mathcal{F}$. 
$\mathcal{F}$
is a $\sigma$-algebra, which is a non-empty collection 
of subsets of $\Omega$ and satisfies the properties above. 
\end{definition} 

Without $P$, the double $(\Omega, \mathcal{F})$ is a 
measurable space, i.e., something which we can
assign a measure to. We will define a measure next. 

\begin{definition}[\emph{Measure}]
A measure is a non-negative countably additive set
function, such that given a $\sigma$-algebra $\mathcal{F}$
and a measure $\mu$, then $\mu:\mathcal{F} \rightarrow 
\mathbb{R}$ such that
\begin{enumerate}
	\item $\mu(A) \geq \mu(\emptyset) = 0 \quad \forall A in \mathcal{F}$.
	\item For $i = 1, 2, \cdots$, if $A_i \in \mathcal{F}$ 
	is a countable
	sequence of disjoint sets, then $\mu(\cup A_i) = \sum \mu(A_i)$.
\end{enumerate}
\end{definition}

\begin{remark}
If $\mu(\Omega) = 1$, then $\mu$ is a probability measure. 
\end{remark}.

From the above definition, we can deduce that a measure
has the following properties. 

\begin{theorem}[\emph{Properties of a measure}]
Suppose $\mu$ is some measure on a measurable space 
$(\Omega, \mathcal{F})$, then $\mu$ satisfies,
\begin{enumerate}
	\item Monotonicity. $A\subset B \Rightarrow P(B) - P(A) - P(B-A) \geq 0$ for $A, B \in \mathcal{F}$.
	\item Subadditivity. For $m = 1, \cdots$, if $A_m \in \mathcal{F}, A \subset \cup_{m=1}^\infty A_m$, then $P(A) \leq \sum_{m=1}^\infty P(A_m)$.
	\item Continuity from below. If $A_i \uparrow A$, i.e,
	$A_1\subset A_2 \subset \cdots$ and $\cup A_i = A$, 
	then $P(A_i) \uparrow P(A)$.
	\item Continuity from above. If $A_i \downarrow A$, 
	then $P(A_i) \downarrow P(A)$. 
\end{enumerate}
\end{theorem}

\begin{proof}
	\textbf{Monotonicity}. Since $B-A = B \cap A^c$ and
	$A$ and $B-A$ are disjoint, then $\mu(A) = \mu(B) + \mu(B-A)$
	and the result follows from the the non-negativity of 
	$\mu$. 
	
	\textbf{Subadditivity}. Let $B_1 = A_1$ and $B_k =
	A_k - \cup_{i=1}^{k-1} A_i$, then $\mu(\cup_{i=1}^\infty A_i) = \sum_{i=1}^\infty \mu(B_i)$,
	 the result follows
	from monotonicity.
	
	\textbf{Continuity from below}. Let $B_k = A_k - A_{k-1}$. Then $\mu(A_k) = \sum_{i=1}^k \mu(B_i)$. Then the 
	result follows. 
	
	\textbf{Continuity from above}. Observe that 
	$A_i^c \uparrow A$, then $P(A_i^c) \uparrow P(A^c)$
	and the result follows.
\end{proof}

To make things a bit more concrete, let us apply this
to an example on the discrete probability space. 

\begin{example}[\emph{Discrete Probability Spaces}]
Let $\Omega$ be a countable set, $\mathcal{F}$ be
the set of all subsets of $\Omega$, for $A \in \mathcal{F}$,
$P(A) = \sum(x \in A) p(x) \geq 0$ and $\sum_{\omega \in \Omega} p(\omega) = 1$. If $\Omega$ is finite, then 
$p(\omega) = 1/|\Omega|$. For example, when
rolling a fair dice, $\Omega = \{1,2,3,4,5,6\}$. 
\end{example}

\begin{example}[\emph{Measure on continuous real line}]
Let $\mathbb{R}$ be the real line, $\mathcal{R}$ be
the Boreal sets (smallest $\sigma$-algebra containing
the open sets), $\lambda$ be the Lebesgue measure (for
$a<b$, $\lambda((a, b]) = b-a$, and $\lambda(\mathbb{R})
= \infty$. To get a probability space, let $\Omega = (0, 1)$,
$\mathcal{F} = \{A \cap (0,1): A \in \mathcal{R}\}$,
and $P = \lambda$. Then $P$ is a Lebesgue measure restricted
to the Borel subsets in interval (0,1). 
\end{example}

\begin{theorem}[\emph{$\sigma$-algebra closed under arbitrary intersection}] \label{1.thm.sigma.closed}
If $\mathcal{F}_i, i \in I$ for an arbitrary (possibly
uncountable) $I$, then $\cap_i \mathcal{F}_i$ is a 
$\sigma$-algebra. 

\end{theorem}

\begin{proof}
	Since $\emptyset, \Omega \in \mathcal{F}_i$, 
	then $\emptyset, \Omega \in \cap \mathcal{F_i}$.
	Furthermore, $A \in \cap \mathcal{F}_i \Rightarrow A \in \mathcal{F}_i$ hence $\cap \mathcal{F}_i$ closed
	under intersection and complement. 
\end{proof}

\begin{definition}[\emph{Smallest $\sigma$-field}]
	Let $\mathcal{A}$ be a collection of subsets of 
	$\Omega$. Then there is a smallest $\sigma$-field
	containing $\mathcal{A}$, written as $\sigma(\mathcal{A})$. 
\end{definition}

\begin{remark}
	The existence of $\sigma(\mathcal{A})$ is
	guaranteed using Theorem \ref{1.thm.sigma.closed}.
	Namely, let $F_i$ be the different $\sigma$-algebra
	that contain $\mathcal{A}$. Then an intersection
	yields the smallest one. 
\end{remark}

\begin{theorem}[\emph{Measure on Product Spaces}]
If $(\Omega_i, \mathcal{F}_i, P_i), i = 1, \cdots, n$ are probability spaces,
let $\Omega = \Omega_1 \times \cdots \times \Omega_n =
\{(\omega_1, \cdots, \omega_n): \omega_i \in \Omega_i\}$,
$\mathcal{F} = \prod_{i=1}^n \mathcal{F}_i$ is
the $\sigma$-field generated by $\{A_1 \times \cdots \times A_n: A_i \in \mathcal{F}_i\}$. Then $P$ is a measure on 
$\mathcal{F}$, given by, 
\[
	P(A_1 \times \cdots \times A_n) = \prod_{i=1}^n P_i(A_i). 
\]
\end{theorem}

We can define a random variable on probability spaces. 
A real valued function $X$ defined on $\Omega$ is a random variable
if for every borel set $B \subset \mathbb{R}$,
\[
	X^{-1}(B) = \{\omega:X(\omega) \in B\} \in \mathcal{F}.
\]

If $X$ is a random variable, then $X$ induces a probability
measure on $\mathbb{R}$ called its distribution, by
setting $\mu(A) = P(X\in A) = P(X^{-1}(A)) =
P(\{\omega:X(\omega) \in B\})$ for every borel set $B \in 
\mathbb{R}$. Equivalently, this means taking $X^{-1}(A) \in \mathcal{F}$ and measuring that set. 

The distribution of $X$ can be described using its
distribution function $F(x) = P(X \leq x)$, which
has the following properties.

\begin{theorem}[\emph{Properties of Distribution Function}]
\label{1.prop.distr}
	All distribution functions $F$ have the following properties,
	\begin{enumerate}
		\item $F$ is non-decreasing.
		\item $\lim_{x\rightarrow\infty} F(x) = 1, \lim_{x \rightarrow - \infty} F(x) = 0$. 
		\item $F$ is right continuous, $\lim_{y\downarrow  x} F(y) = F(x)$.
		\item $F(x-) = \lim_{y \uparrow x} F(y) = P(X>x)$.
		\item $P(X=x) = F(x) - F(X-)$. 
	\end{enumerate}
\end{theorem} 

\begin{proof}
\begin{enumerate}
	\item Suppose $x < y$ then $\{X < x\} \subset \{X < y\}$
	therefore $F(x) \leq F(y)$. 
	\item $\lim {x\rightarrow\infty} \{X < x\} = \Omega$ 
	and $\lim {x\rightarrow-\infty} \{X < x\} = \emptyset$.
	\item  $\lim_{y\downarrow  x} \{X < y\} = \{X < x\}$. 
	\item  $\lim_{y\uparrow  x} \{X < y\} = \{X < x\}$.
	\item Since $\{X = x\} = \{X \geq x\} \cap \{X < x\}^c$,
	then taking limits on both sides using 3, 4 gives the 
	result.  
\end{enumerate}
\end{proof}

\begin{corollary}
	If $F$ satisfies 1, 2, 3 of Theorem \ref{1.prop.distr}, then
	it is a distribution function of some random variable. 
\end{corollary}
\begin{proof}
	
\end{proof}
\end{document}